\documentclass[../writeup.tex]{subfiles}

\begin{document}  



\section{The Two-Bucket Camera}

\subsection{Notations}

The coded two-bucket (C2B) camera is a pixel-wise coded exposure camera that outputs two images in a single exposure.\cite{weiCodedTwoBucketCameras2018} Each pixel in the sensor has two photo-collecting site, i.e. the two \textit{buckets}, as well as a 1-bit writable memory controlling which bucket is actively collecting light. It was shown previously that C2B camera is capable of one-shot 3D reconstruction by solving a simpler image demosaicing and illumination demultiplexing problem instead of a difficult 3D reconstruction problem. We summarize the following notations relevant to discussion

\begin{table}[!htbp]
    \begin{center}
    \begin{tabular}{rll}
        \multicolumn{1}{r}{\bf} & \multicolumn{1}{l}{\bf Notation}   &\multicolumn{1}{l}{\bf Meaning}\\
        \hline \\
                             & F                                       & number of video frames \\
                             & P                                       & number of pixels \\
                             & S                                       & number of sub-frames \\
                             & h,w                                     & dimension of image \\
        $P\times F\times S$  & $\bC$                                   & code tensor \\
        $P\times 1\times S$  & $\widetilde{\bC}$                       & 1-frame code tensor that spatially multiplex $F$ frame tensor $\bC$ \\
        $F\times S$          & $\bC^p$                                 & activity of bucket 0 pixel $p$ cross all frames and sub-frames \\
        $F\times S$          & $\overline{\bC}^p$                       & activity of bucket 1 pixel $p$ cross all frames and sub-frames \\
        $1\times S$          & $\bc^p_f$                               & active bucket of pixel $p$ in the sub-frames of frame $f$ \\
        $1\times L$          & $\bl_s$                                 & scene's illumination condition in sub-frame $s$ of every frame \\
        $P\times S$          & $\bC_f=[\bc_1^p;\cdots;\bc_F^p]$        & activity of bucket activity of all pixels across all sub-frames of $f$ \\
        $S\times L$          & $\bL= [\bl_1;\cdots;\bl_S]$             & time-varying illumination condition (same for all frames) \\
        $2F\times S$         & $\bW$                                   & optimal bucket multiplexing matrix \\
                             & $\bt^p$                                 & transport vector at pixel $p$ \\
        $F \times 1$         & $\bi^p,\hat{\bi}^p$                     & measured two-bucket intensity at pixel $p$ in $F$ frames \\
        $F \times 1$         & $r,\hat{r}$                             & illumination ratios at pixel $p$ in $F$ frames \\
        $F\times P$          & $\bI = [\bi^1 \cdots \bi^P],\hat{\bI}$  & two-bucket image sequence in $F$ frames \\ 
        $P\times 2F$         & $\sI = [\bI^T \;\hat{\bI}^T]$           & two-bucket image sequence \\
        $P\times 2$          & $\bY$                                   & two-bucket illumination mosaic \\
        $S\times 1$          & $\si^p$                                 & pixel intensity under $S$ illuminations at pixel $p$ \\
        $P\times S$          & $\bX = [\si^1 \cdots \si^P]^T$          & pixel intensity under $S$ illuminations \\    
        $2P\times 1$         & $\by = \vec{\bY}$                       & vectorized two-bucket illumination mosaic \\
        $SP\times 1$         & $\bx = \vec{\bX}$                       & vectorized pixel intensity under $S$ illumiantions \\
        $2P\times 2PF$       & $\bB$                                   & subsampling linear map \\
        $2P\times SP$        & $\bA = \bB(\bW\otimes \bI_P)$           & illumination multiplexing and subsampling linear map \\
    \end{tabular}
    \end{center}
\end{table}
\noindent Illumination ratios are albedo \textit{quasi-invariant}, a property which can be exploited for downstream processing
\[
    r = \frac{\bi^p[f]}{\bi^p[f] + \hat{\bi}^p[f]} 
    \quad\quad
    \hat{r} =   \frac{\hat{\bi}^p[f]}{\bi^p[f] + \hat{\bi}^p[f]} 
\]

\begin{figure}[h!]
    \begin{center}
        \includegraphics[width=5in]{image_formation_schema}
    \end{center}
    \caption{Image Formation Sketch}
    \label{fig:image_formation_schema}
\end{figure}

\subsection{The Forward Model}

\paragraph{Subsampling Mapping}
Let $\bS \in \{1,2,\cdots,F\}^P$ be a vector specifying how the one-frame code tensor $\widetilde{\bC}$ is constructed, i.e. $\tilde{\bc}_1^{p} := \bc_{\bS_p}^p$, for all pixels $p$. We can view $\bS$ as a mask to construct a \textbf{S}ubsampling linear map that maps vectorized two-bucket image sequences $\sI$ to the vectorized illumination mosaics $\bY$. In particular, let $\bB' \in \R^{P\times PF}$ and $\bB \in \R^{2P \times 2PF}$ be defined as follows 
\begin{align*}
    \bB' &=
    \begin{bmatrix}
        \diag{\mathbbm{1}_{\{1\}}(\bS) } & \diag{\mathbbm{1}_{\{2\}}(\bS) } & \cdots & \diag{\mathbbm{1}_{\{F\}}(\bS) }
    \end{bmatrix} \\
    \bB &=  \bI_2 \otimes \bB' = 
    \begin{bmatrix}
        \bB' & \mathbf{0} \\
        \mathbf{0} & \bB' \\
    \end{bmatrix}
\end{align*}
Then we have the following relation between $\sI$ and $\bY$,
\begin{equation}
    \label{eq:subsampling_relation}
    \vec{\bY} = \bB \vec{\sI}
\end{equation}
In essence, $\bB$ is a linear operator that trade spatial resolution (measures $\frac{1}{F}$ of the pixels for each frame) for temporal resolution (one two-bucket shot instead of acquiring $F$ frames). We can think of a parallel in RGB color imaging, where bayer mosaic trade spatial resolution for spectral resolution. As an example when $F=3$ and $P=4$, the corresponding $\bS$, when reshaped to dimension of a $2\times 2$ image, and single image subsampling linear map $\bB'$ are given by
\[
    \bS = 
    \begin{bmatrix}
        1 & 2 \\
        2 & 3
    \end{bmatrix}    
    \qquad
    \bB' = 
    \begin{bmatrix}
        1& 0& 0& 0& 0& 0& 0& 0& 0& 0& 0& 0 \\
        0& 0& 0& 0& 0& 1& 0& 0& 0& 0& 0& 0 \\
        0& 0& 0& 0& 0& 0& 1& 0& 0& 0& 0& 0 \\
        0& 0& 0& 0& 0& 0& 0& 0& 0& 0& 0& 1 \\
    \end{bmatrix}
\]

\paragraph{Image Formation}
Per-pixel image formation model is
\[
    \begin{bmatrix}
        \bi^p \\ \hat{\bi}^p
    \end{bmatrix}
    = 
    \begin{bmatrix}
        \bC^p \\ \overline{\bC}^p
    \end{bmatrix}
    \begin{bmatrix}
        \bl_1 \bt^p \\ \vdots \\ \bl_S \bt^p
    \end{bmatrix}
    = 
    \begin{bmatrix}
        \bC^p \\ \overline{\bC}^p
    \end{bmatrix}
    \si^p
\]
If bucket activity is same for all pixels and we use the optimal bucket multiplexing matrix $\bW$, we can write the above linear relationship compactly for all pixels as
\begin{equation}
    \label{eq:image_formation}
    \sI = \bX \bW^T
\end{equation}

\paragraph{Linear Model} As shown in Figure~\ref{fig:image_formation_schema}, illumination multiplexing and spatial subsampling can be combined to obtain a single linear function that maps images under $S$ different illuminations $\bX$ to the two-bucket images $\bY$. From (\ref{eq:subsampling_relation}) and (\ref{eq:image_formation}), there exists a linear relationship between $\bx$ and $\by$, 
\begin{equation}
    \label{eq:linear_mapping}
    \by = \bB \vec{\sI} = \bB \vec{\bX \bW^T} = \bB(\bW \otimes \bI_P) \vec{\bX} = \bA\bx
\end{equation}
where $\bA \in \R^{2P\times SP}$ be a linear map that illumination multiplexes and subsamples $\bX$,
\[
    \bA = \bB(\bW\otimes \bI_P)
\]
and $\bI_P\in\R^{P\times P}$ is identity. 

\subsection{Image Processing Pipeline}
The reconstruction pipeline is as follows
\begin{enumerate}
    \item Use $\widetilde{\bC}$ for bucket activities and capture the two-bucket image $\bY$
    \item upsample the images to full resolution images $\sI$
    \item demultiplex $\sI$ to obtain $S$ full resolution images $\bX$ as a least squares solution to a (\ref{eq:image_formation})
    \item use $\bX$ to solve for disparity and albedo
\end{enumerate}
Step 2 and 3 are critical to downstream reconstuctions. When $S=3,S=4$ and $\bS$ being analogous to bayer mask, we can upsample the images using standard demosaicing algorithms. However, it is not immediately obvious to extend demosaicing methods to support arbitrary $\bS$, or more specifically, for scenarios where the spatial subsampling scheme is not bayer and when number of frames is not 3. 





\end{document}